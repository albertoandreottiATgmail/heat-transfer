%\documentstyle{article}
\documentclass[letter,11pt,english]{article}
\special{papersize=8.5in,12in}

\usepackage{ifsym}
\usepackage[colorlinks=true, urlcolor=blue, dvipdfm, hypertex]{hyperref}
\oddsidemargin=.15in
\evensidemargin=.15in
\textwidth=6in
\topmargin=-.5in
\textheight=9in
\parindent=0in
\pagestyle{plain}
%Conditional compilation statement for professional version.
\usepackage{ifthen}
\newboolean{DEBUG}
\setboolean{DEBUG}{true}

\begin{document}

\pagestyle{headings}
\setcounter{page}{1}
\pagenumbering{arabic}
\begin{center}
{\Large Jose Pablo Alberto Andreotti} \\[.5pc]
(+54) 351-4730292 $\;$ albertoandreotti\verb|@|gmail.com $\;$ (+54) 351-155937792 \\[3pc]
\end{center}
{\large \bf Personal Information} \\*[-.8pc]
\underline{\hspace{6in}} \\
\\
{\bf Nationality}, Argentinean/Italian.\\
{\bf Address}, 369, Zelaya Street, C\'ordoba. Argentina.\\

{\large \bf Education} \\*[-.8pc]
\underline{\hspace{6in}} \\
{\bf Postgradute Specialization in Distributed Systems}\\
\href{http://www.famaf.unc.edu.ar/}{Faculty of Mathematics, Astronomy and Physics}, National University of C\'ordoba, 2009. \\
GPA: 8.87/10. \\
Final Project: \href{https://docs.google.com/viewer?a=v&pid=explorer&chrome=true&srcid=0B5AOpwg8IzVANjJlODZhZDctNWUzMS00MmNhLWI3OWMtMWNhMTdjODQwNjVl&hl=en}{High Scalable Scientific Computing using Hadoop}.\\
%Final Project: I worked on solving scientific computing problems using Hadoop, a description can be found \href{https://docs.google.com/viewer?a=v&pid=explorer&chrome=true&srcid=0B5AOpwg8IzVANjJlODZhZDctNWUzMS00MmNhLWI3OWMtMWNhMTdjODQwNjVl&hl=en}{here}.\\
\\
{\bf Computer Engineer}\\
Faculty of Exact, Physical and Natural Sciences, National University of C\'ordoba, 2007.\\
GPA: 8.14/10. \\
Thesis: \href{https://docs.google.com/viewer?a=v&pid=explorer&chrome=true&srcid=1gdJXYgQtLDHDOxGDtbKzdmAl1LmNx-yo4w6vNl-K_Z-1YocLhtJxMvoqGvd1&hl=en}{Building a dual core JVM for embedded realtime systems.}\\

{\large \bf Professional Experience}\\*[-.8pc]
\underline{\hspace{6in}}
\\
\begin{tabular}{ p{2cm} l }
  {\bf Current} & {\bf Samsung Strategy and Innovation Center, through BTS Software.}\\
   June /2014   & $\triangleright$ At BTS we work together with the Samsung Strategy and Innovation\\
   to present   & Center (SSIC) building the \href{https://www.samsungsami.io/}{SAMI} Big Data platform.\\
                & $\triangleright$ Sr. Software Engineer with focus on scalability and performance.\\ 
\\
  {\bf Past}    & {\bf \href{http://www.intel.com}{Intel Corp}, Sr. Software Engineer.}\\
   Jan./2011     & $\triangleright$ Worked in Intel's \href{https://software.intel.com/en-us/context-sensing-sdk}{Context Awareness and  Inference Engine}.\\
   June/2014     & $\triangleright$ Built a scalable knowledge representation database used for user\\               
                & recommendations.\\
                & $\triangleright$ Developed Web Services and Embedded platform software.\\
\\     
   June/2014     & {\bf \href{http://www.nimbuzz.com/en/about}{Nimbuzz}, Software Engineer.}\\
   Jan./2011   	&  $\triangleright$ Software development of the Nimbuzz VoIP client for iOS.\\ 
\\
   2011         & {\bf Member of the Computer Engineering Council at National} \\
		& {\bf University of C\'ordoba.}\\ 
		& $\triangleright$ Reviewer of Study plans and Curricula.\\
\\
   Jun 2008     & {\bf Software Engineer at Motorola Argentina}\\
   Jun 2010     & $\triangleright$ Development of Embedded Software in C/C++ for Cable Modem products.\\
		& $\triangleright$ Development of tests for the Motorola \href{http://www.motorola.com/web/Business/_Documents/White%20Paper/_Static%20files/NBBS%20WiMAX%20White%20Paper%20557127-001-b.pdf}{NBBS platform}.\\
		& $\triangleright$ Developed and maintained \href{http://en.wikipedia.org/wiki/Advanced_Telecommunications_Computing_Architecture}{AdvancedTCA} clusters running Linux.\\
\\
\end{tabular}

\begin{tabular}{ p{2cm} l }

   2007/2008    & {\bf Self-employed}\\
		& Optimization of Scientific Workloads using Parallel Programming Techniques.\\
\\
   Mar./2006    & {\bf Computer Architecture Laboratory, UNC}\\
   Oct./2007    & Research Internship.\\
		&$\triangleright$ Optimization of Scientific programs over multiprocessor architectures and\\
		& parallel programming with MPI, HPF and OpenMP.\\
		&$\triangleright$ Testing of Embedded Systems for the Biomedical Industry.\\
\\
   2007         &{\bf Testing of the JOP processor and JDK}\\
		& I worked testing the \href{http://jopdesign.com}{Java Optimized Processor (JOP)}.
\\
\end{tabular}					
\\


{\large \bf Programming Experience} \\*[-.8pc]
\underline{\hspace{6in}} \\
$\triangleright$ Develpment of Embedded Software in C/C++, and Java.\\
$\triangleright$ Experience on mobile platforms such as Android and iOS.\\
$\triangleright$ Experience with RESTful Web Services on Python/Django and Scala/Akka/Play.\\
$\triangleright$ Low level programming of Digital Signal Processors in C/Assembly.\\



{\large \bf Additional Coursework}\\ *[-.8pc]
\underline{\hspace{6in}}
$\triangleright$ \href{http://cs.famaf.unc.edu.ar/wiki/materias/pln}{Natural Language Processing and Text Mining}, Franco Luque y Laura Alonso, Fa.M.A.F, National University of C\'ordoba, 2015.\\
$\triangleright$ \href{http://aprendizajengrande.net/}{Machine Learning over big data sets},  \href{http://duboue.net/}{Pablo Duboue}, Fa.M.A.F, National University of C\'ordoba, August/November of 2014.\\
$\triangleright$ Introduction to Data Science, University of Washington through Coursera. First Semester
of 2014\\
$\triangleright$ \href{https://www.coursera.org/course/progfun}{Functional Programming Principles in Scala},  Martin Odersky. Ecole Polytechnique Federale de Lausanne, December of 2013.  \\
$\triangleright$ \href{https://www.coursera.org/course/images}{Image and video processing: From Mars to Hollywood with a stop at the hospital}, Guillermo Sapiro, Duke University, first semester of 2013. \\
$\triangleright$ \href{https://www.coursera.org/course/dsp}{Digital Signal Processing}, Paolo Prandoni and Martin Vetterli, Ecole Polytechnique Federale de Lausanne, first semester of 2013.\\
$\triangleright$ \href{http://www.ml-class.org}{Machine Learning}. \footnote{This is the online version of the course.}, Andrew Ng, Stanford University, first semester of 2012. \\
$\triangleright$ High Performance Computing: Models, Methods and Means\footnote{Unstructured Postgraduate,
and Graduate Course, Fa.M.A.F. Sincronized with Prof. Thomas Sterling's 
\href{https://www.cct.lsu.edu/csc7600/Home.html}{CSC7600}.}, first semester 2010.\\
$\triangleright$ The Semantic Web, PhD Jorge Cardoso, SAP Research Germany, first semester of 2009.\\
$\triangleright$ Introduction to Radar Signal Processing, Msc Oscar Bria, 
\href{http://www.invap.net/index-e.php}{INVAP}, second semester of 2009. \href{http://postgrado.info.unlp.edu.ar/Cursos/Cursos/11-2011_Introduccion_al_Procesamiento_de_Senales_Radar.pdf}{Syllabus}.\\
$\triangleright$ Performance and Scaling in E-Commerce Systems WS, Alex Buchmann, second semester of 2008. 
\href{http://www.dvs.tu-darmstadt.de/teaching/perf/2008/}{Syllabus}.\\

\newpage
{\bf Publications} \\*[-.8pc]
\underline{\hspace{6in}} \\
$\triangleright$ Infrastructure for enabling fast machine learning application prototyping in the cloud. Intel SWPC, Guadalajara, MX, 2013.\\
$\triangleright$ I contributed to \href{https://github.com/rasmusbergpalm/DeepLearnToolbox} {DeepLearnToolbox},
an open source Deep Learning toolbox written in Matlab. I added features to the CNN\footnote{Convolutional Neural Network.}
part of the toolbox and added an example for Spoken Language Classification\footnote{Based on the research found in
\href{http://research.microsoft.com/en-us/um/people/dongyu/nips2009/papers/montavon-paper.pdf} {this paper}} \\
$\triangleright$ Recommendations for Movies using Distributed Pattern Matching. Intel SWPC, C\'ordoba, AR, 2013.\\
$\triangleright$ Optimization of an hydrodynamical bi-dimensional model. ENIEF, Congress on Numerical Methods 
and their Applications, 2009, \href{http://www.cimec.org.ar/ojs/index.php/mc/article/viewFile/2930/2867}{link}. \\
$\triangleright$ Using JOP to build a chip multiprocessor JVM for embedded realtime systems. Annals of CACIC, 2007.\\

{\large \bf Conferences attended as a lecturer} \\*[-.8pc]
\underline{\hspace{6in}} \\
{\bf  Intel Software Professional Conference, 2013.}\\
Title: Prototyping Machine Learning Apps in the Cloud. 

{\bf  ENIEF, Congress on Numerical Methods and their Applications, 2009.}\\
Title: Optimization of an hydrodynamical bidimensional model. 
\\
{\bf  Second Open Conferences in Computer Engineering, National University of C\'ordoba, 2008.}\\
Title: Parallel Programming.
\\
{\bf  First Open Conferences in Computer Engineering, National University of C\'ordoba, 2007.}\\
Title: A multiprocessor JVM based on JOP.
\\
{{\bf  XIII Argentinian Congress on Computer Science, 2007.}\\
Article's Title: Using JOP to build a chip multiprocessor JVM for embedded real-time systems.}\\
\\


{\bf Scholarships and Awards} \\*[-.8pc]
\underline{\hspace{6in}} \\
\renewcommand{\labelitemi}{}
$\triangleright$ Distinction to the best average grade.
Given by the College of Engineering of C\'ordoba, 2007.\\
$\triangleright$ Grant earned for the completion of the Thesis Project.
Given by the Science Agency of C\'ordoba, 2006.\\


{\large \bf Teaching Experience} \\*[-.8pc]
\underline{\hspace{6in}} \\
{\bf Concurrent Programming TA}\\
Computer Engineering faculty, National University of Cordoba, 2004.
\\
{\bf Multivariate Calculus TA}\\
Electronic and Computer Engineering School, National University of Cordoba, 2003.
\\
\newpage
{\large \bf Language Skills} \\*[-.8pc]
\underline{\hspace{6in}} \\
{\bf Spanish:} Native.\\
{\bf German:} Intermediate.\\
{
{\bf English}\\
$\triangleright$ TOEFL iBT. Outcome 109 out of 120, November 2010.\\
$\triangleright$ TOEFL iBT. Outcome 110 out of 120, March 2008.\\
$\triangleright$ First Certificate Examination grade B. Given by Cambridge University, 2000.\\
{\bf Italian:} Advanced.\\
$\triangleright$ Four years course at the School of Languages of the National University of C\'ordoba, 2008.
\\
{\bf French:} Basic reading proficiency.\\
\end{document}




