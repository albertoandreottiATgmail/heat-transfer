%\documentstyle{article}
\documentclass[letter,11pt,english]{article}
\special{papersize=8.5in,12in}

\usepackage{ifsym}
\usepackage[dvipdfm, hypertex]{hyperref}
\oddsidemargin=.15in
\evensidemargin=.15in
\textwidth=6in
\topmargin=-.5in
\textheight=9in
\parindent=0in
\pagestyle{plain}
%Conditional compilation statement for professional version.
\usepackage{ifthen}
\newboolean{DEBUG}
\setboolean{DEBUG}{true}

\begin{document}

\pagestyle{headings}
\setcounter{page}{1}
\pagenumbering{arabic}

\begin{center}
{\Large Jose Pablo Alberto Andreotti} \\[.5pc]
(+54) 351-4730292 $\;$ albertoandreotti\verb|@|gmail.com $\;$ (+54) 351-155937792 \\[3pc]
\end{center}
{\large \bf Personal Information} \\*[-.8pc]
\underline{\hspace{6in}} \\
\\
{\bf Nationality}, Argentinian/Italian.\\
\\
{\bf Birth date}, October 8, 1982. \\
\\
{\bf Birth place}, Pico Truncado, Santa Cruz Provice, Argentina.\\
\\
{\bf Address}, 552, Emilio Castelar Street, C\'ordoba. Argentina.\\

{\large \bf Education} \\*[-.8pc]
\underline{\hspace{6in}} \\
{\bf Distributed Systems, MSc.}\\
\href{http://www.famaf.unc.edu.ar/}{Famaf}, University of C\'ordoba, 2009. \\
Average Grade: 8.87/10. \\
Final Project: High Scalable Scientific Computing using Hadoop, found \href{https://docs.google.com/viewer?a=v&pid=explorer&chrome=true&srcid=0B5AOpwg8IzVANjJlODZhZDctNWUzMS00MmNhLWI3OWMtMWNhMTdjODQwNjVl&hl=en}{here}.\\
%Final Project: I worked on solving scientific computing problems using Hadoop, a description can be found \href{https://docs.google.com/viewer?a=v&pid=explorer&chrome=true&srcid=0B5AOpwg8IzVANjJlODZhZDctNWUzMS00MmNhLWI3OWMtMWNhMTdjODQwNjVl&hl=en}{here}.\\
\\
{\bf Computer Engineer, MSc.\footnote {This degree is the equivalent to the European Master level, it is obtained
through 5 years of course work and 1 year of research.The Argentinian name is "t\'itulo de grado".}}\\
Engineering Faculty, University of C\'ordoba, 2007.\\
Average Grade: 8.14/10. \\
Thesis: Building a dual core JVM for embedded realtime systems, found \href{https://docs.google.com/viewer?a=v&pid=explorer&chrome=true&srcid=1gdJXYgQtLDHDOxGDtbKzdmAl1LmNx-yo4w6vNl-K_Z-1YocLhtJxMvoqGvd1&hl=en}{here}.\\
%Thesis: I developed a dual core JVM, which was implemented on an FPGA, a short description of the project can be accessed \href{https://docs.google.com/viewer?a=v&pid=explorer&chrome=true&srcid=1gdJXYgQtLDHDOxGDtbKzdmAl1LmNx-yo4w6vNl-K_Z-1YocLhtJxMvoqGvd1&hl=en}{here}.\\


{\bf High School at Colegio Coraz\'on de Mar\'ia}\\
Management Oriented School, 2000.\\
Average Grade: 8.22/10.\\


{\large \bf Additional Coursework}\\ *[-.8pc]
\underline{\hspace{6in}}
$\triangleright$ Functional Programming Principles in Scala,  Martin Odersky. Ecole Polytechnique Federale de Lausanne, December of 2013. Link \href{https://www.coursera.org/course/progfun}{here}. \\
$\triangleright$ Image and video processing: From Mars to Hollywood with a stop at the hospital, Guillermo Sapiro, Duke University, first semester of 2013. Link \href{https://www.coursera.org/course/images}{here}. \\
$\triangleright$ Digital Signal Processing, Paolo Prandoni and Martin Vetterli, Ecole Polytechnique Federale de Lausanne, first semester of 2013. Link \href{https://www.coursera.org/course/dsp}{here}. \\
$\triangleright$ Machine Learning\footnote{This is the online version of the course.}, Andrew Ng, Stanford University, first semester of 2012. Link \href{http://www.ml-class.org}{here}. \\
$\triangleright$ High Performance Computing: Models, Methods and Means\footnote{Unstructured Postgraduate,
and Graduate Course, Fa.M.A.F. Sincronized with Prof. Thomas Sterling's 
\href{https://www.cct.lsu.edu/csc7600/Home.html}{CSC7600}.}, first semester 2010.\\
$\triangleright$ The Semantic Web, PhD Jorge Cardoso, SAP Research Germany, first semester of 2009.\\
$\triangleright$ Introduction to Radar Signal Processing, Msc Oscar Bria, 
\href{http://www.invap.net/index-e.php}{INVAP}, second semester of 2009. \href{http://postgrado.info.unlp.edu.ar/Cursos/Cursos/11-2011_Introduccion_al_Procesamiento_de_Senales_Radar.pdf}{Syllabus}.\\
$\triangleright$ Performance and Scaling in E-Commerce Systems WS, Alex Buchmann, second semester of 2008. 
\href{http://www.dvs.tu-darmstadt.de/teaching/perf/2008/}{Syllabus}.\\


{\bf Publications} \\*[-.8pc]
\underline{\hspace{6in}} \\
$\triangleright$ Infrastructure for enabling fast machine learning application prototyping in the cloud. Intel SWPC, Guadalajara, MX, 2013.\\
$\triangleright$ I contributed to \href{https://github.com/rasmusbergpalm/DeepLearnToolbox} {DeepLearnToolbox},
an open source Deep Learning toolbox written in Matlab. I added features to the CNN\footnote{Convolutional Neural Network.}
part of the toolbox and added an example for Spoken Language Classification\footnote{Based on the research found in
\href{http://research.microsoft.com/en-us/um/people/dongyu/nips2009/papers/montavon-paper.pdf} {this paper}} \\
$\triangleright$ Recommendations for Movies using Distributed Pattern Matching. Intel SWPC, C\'ordoba, AR, 2013.\\
$\triangleright$ Optimization of an hydrodynamical bi-dimensional model. ENIEF, Congress on Numerical Methods 
and their Applications, 2009, \href{http://www.cimec.org.ar/ojs/index.php/mc/article/viewFile/2930/2867}{link}. \\
$\triangleright$ Using JOP to build a chip multiprocessor JVM for embedded realtime systems. Annals of CACIC, 2007.\\
%$\triangleright$ Optimization of a bidimensional hydrodynamic model. ENIEF, 2009.\\


{\large \bf Conferences attended as a lecturer} \\*[-.8pc]
\underline{\hspace{6in}} \\
{\bf  Intel Software Professional Conference, 2013.}\\
Title: Prototyping Machine Learning Apps in the Cloud. 

{\bf  ENIEF, Congress on Numerical Methods and their Applications, 2009.}\\
Title: Optimization of an hydrodynamical bidimensional model. 
\\
{\bf  Second Open Conferences in Computer Engineering, National University of C\'ordoba, 2008.}\\
Title: Parallel Programming.
\\
{\bf  First Open Conferences in Computer Engineering, National University of C\'ordoba, 2007.}\\
Title: A multiprocessor JVM based on JOP.
\\
{{\bf  XIII Argentinian Congress on Computer Science, 2007.}\\
Article's Title: Using JOP to build a chip multiprocessor JVM for embedded real-time systems.}\\
\\


{\bf Scholarships and Awards} \\*[-.8pc]
\underline{\hspace{6in}} \\
\renewcommand{\labelitemi}{}
$\triangleright$ Distinction to the best average grade.
Given by the College of Engineering of C\'ordoba, 2007.\\
$\triangleright$ Grant earned for the completion of the Thesis Project.
Given by the Science Agency of C\'ordoba, 2006.\\

\newpage
{\large \bf Professional Experience}\\*[-.8pc]
\underline{\hspace{6in}}
\\
\begin{tabular}{ p{3cm} l }
  {\bf 2010 to present} & {\bf Sr. Software Engineer at \href{http://www.intel.com}{Intel Corp}.}\\
                        & 	$\triangleright$ Worked in a Context Awareness and Inference Engine.\\ 
                        & 	$\triangleright$ Worked in a project for Data Mining as a Service.\\
                        & 	$\triangleright$ Development of Web Services and Embedded platform software.\\
%  {\bf 2010 to 2010}    & 	{\bf Software Engineer at \href{http://www.nimbuzz.com/en/about}{Nimbuzz}.}\\
  {\bf 2010}            & 	{\bf Member of the Computer Engineering Council at National}\\
                        &   {\bf University of C\'ordoba}.\\
  {\bf 2008 to 2010}    & 	{\bf Software Engineer at Motorola.}\\
                        & 	$\triangleright$ Development of telecommunications software in C/C++.\\ 
  {\bf 2007}            & 	{\bf Self-employed in Software Optimization.}\\
                        &	I worked in projects involving the optimization of scientific\\
                        &	applications in multiprocessor systems.\\            
  {\bf 2007}            & 	{\bf Testing of the JOP processor and JDK.}\\
                        & 	I worked testing JOP(Java Optimized Processor) for Dr. Martin\\
                        &   Schoeberl from TU Wien.\\
  {\bf 2006 to 2008}    & 	{\bf Research Internship, Computer Architecture Lab, UNC.}\\
                        & 	$\triangleright$ Optimization of Scientific programs over computer clusters and\\
                        & 	parallel programming with MPI, HPF and OpenMP.\\
                        & 	$\triangleright$ Software development and testing in C/C++.\\
                        
			
\end{tabular} 
\\

{\large \bf Programming Experience} \\*[-.8pc]
\underline{\hspace{6in}} \\
$\triangleright$ Develpment of embedded Software in C/C++, and Java.\\
$\triangleright$ Experience on mobile platforms such as Android and iOS.\\
$\triangleright$ Experience with RESTful Web Services on Python/Django and Scala/Akka/Play.\\
$\triangleright$ Low level programming of Digital Signal Processors in C/Assembly.\\

{\large \bf Teaching Experience} \\*[-.8pc]
\underline{\hspace{6in}} \\
{\bf Concurrent Programming TA}\\
Computer Engineering faculty, National University of Cordoba, 2004.
\\
{\bf Multivariate Calculus TA}\\
Electronic and Computer Engineering faculty, National University of Cordoba, 2003.
\\

{\large \bf Languages} \\*[-.8pc]
\underline{\hspace{6in}} \\
{\bf Spanish}\\
$\triangleright$ Native Spanish speaker.
\\
{\bf German}\\
$\triangleright$ Currently attending the third year of a four year course in German 
at the School of Languages of the National University of C\'ordoba.
 \\
{
{\bf English}\\
$\triangleright$ TOEFL iBT. Outcome 109 out of 120, November 2010.\\
$\triangleright$ TOEFL iBT. Outcome 110 out of 120, March 2008.\\
$\triangleright$ First Certificate Examination grade B. Given by Cambridge University, 2000.\\
$\triangleright$ Several courses attended from 1992 to 2000.\\
{\bf Italian}\\
$\triangleright$ Four years course at the School of Languages of the National University of C\'ordoba, 2008.
\\
{\bf French}\\
$\triangleright$ One year course at "Alliance Fran\c caise C\'ordoba", 2000.

\end{document}




