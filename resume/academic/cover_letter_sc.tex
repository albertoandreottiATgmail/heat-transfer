%\documentstyle{article}
\documentclass[letter,12pt,english]{article}
\special{papersize=8.5in,12in}

\usepackage{ifsym}
\usepackage[dvipdfm, hypertex]{hyperref}
\oddsidemargin=.15in
\evensidemargin=.15in
\textwidth=6in
\topmargin=-.5in
\textheight=9in
\parindent=0in
\pagestyle{plain}


\begin{document}




{ \hfill \today \\ \\ 



{
\raggedright{}
Mobile Communications Group \\
Intel Corp
}
\\
\\
\large
\bf Cover Letter} \\*[-.8pc]
\\
\\
I hold a Computer Engineering degree and a Distributed Systems degree from the Engineering Faculty
and the Mathematics Faculty respectively, both at the National University of C\'ordoba, Argentina.
I would like to be considered for a position as a Software Engineer at the Intel's Mobile Communications Group
organization.
\\
My first work experience was at the Computer Architecture Laboratory of my faculty. I worked for two
years there, and I focused on performance optimization of large scale finite element simulations of 
hydraulic systems. During this time, I also conducted work for my graduate thesis about the design
of an embedded multi-core Java processor.\\
Using vectorization, OpenMP, and MPI programming techniques I optimized the running time of an 
hydrodynamical model which was used by civil engineers to predict the effect of constructing a bridge
over the Paran\'a
\footnote{The Paran\'a is the second-longest river in South America, and the mentioned bridge
will be the biggest bridge in the country once it's construction is finished during this year. Please
check my resume for a link to a publication describing this work} River,
in Argentina. \\
My Master thesis solved the problem of building a Java Virtual Machine as a multicore system. At that time
multicore processors were an emerging technology so my contribution was to propose solutions to 
problems such as Scheduling of Java threads, Synchronization, and Garbage Collection across multiple
processors. \\
After completing my Masters thesis I left the Computer Architecture Laboratory and pursued a career
in software development in industry. The first position I held was in Motorola Argentina, as an embedded
software developer. I focused on software for communication systems such as Soft Switches that are built
as a cluster of special purpose boards. I acquired valuable experience in software development in general,
working with big amounts of code, collaborating with a team of people and planing work to meet deadlines.
During this same period of time, I started to pursue a degree in Distributed Systems at the Faculty of 
Mathematics at National University of C\'ordoba.
\\
After completing my second degree and acquiring more experience as an engineer I decide to look
for positions more challenging on the technical side. The work at Motorola had become too tedious
in the sense that too much attention was put by management on artificial quality goals rather
that in actually trying to create better products.
\\
In my search for real engineering challenges I joined Intel. I consider projects at Intel to be truly ambitious,
and to really focus on delivering value. My work at Intel has been mainly on Context Awareness and 
CSP Services. \\
The Context Awareness project was about providing APIs to developers of mobile apps
so they can access high level states such as the physical activity or the user 
location\footnote{The high level location would be a value in the set \{home, work, other\}, and the low
level raw sensor data would include for example the GPS coordinates}.
This involved the development of a client engine to perform collection of raw sensor data such
as accelerometers, gps, cameras and microphones. I led the development of the Android version of this client
which involved both Java and C++\footnote{C++ coding was supported by Android's NDK.} programming.\\
My work in CSP Services was divided into two parts; the Choices Service and the Recommender Service.
The Recommender Service focused on the development of a large scale graph processing database which 
was used as a Knowledge Representation engine to allow the real time processing of user recommendations in 
the form of graph queries. We used the Scala programming language and the Akka framework to perform
this development.
\\
The work in the Choices Service involved the development of a Rest Web Service, using Python and Django. 
The mission of this service was to collect POIs\footnote{POIs: points of interest.} information from
different providers such as Yelp or Factual and to provide a normalized data access interface to developers.\\
Finally, I also took part of some POC development related to the CSP Context Sensing Service. I developed
two demos exploring novel features of HTML5 involving audio processing and written character recognition.\\
I thank you for taking the time to consider my application and I look forward to the opportunity of further
discussing with you details of the position and the project. Should you have any question about my previous experience,
please do not hesitate to contact me.
\\

Sincerely,
\\
\\
Jose P. Alberto Andreotti.

\end{document}




