%\documentstyle{article}
\documentclass[letter,12pt,english]{article}
\special{papersize=8.5in,12in}

\usepackage{ifsym}
\usepackage[dvipdfm, hypertex]{hyperref}
\oddsidemargin=.15in
\evensidemargin=.15in
\textwidth=6in
\topmargin=-.5in
\textheight=9in
\parindent=0in
\pagestyle{plain}


\begin{document}

{ \hfill \today \\ \\ 

\large
\bf Cover Letter} \\*[-.8pc]
\\
\\
I hold a Computer Engineering degree and a Distributed Systems degree from the Engineering Faculty
and the Mathematics Faculty respectively, both at the National University of C\'ordoba, Argentina.
I would like to be considered for a position as a Senior Software Engineer in Machine Learning at HP.
\\
My first work experience was at the Computer Architecture Laboratory of my faculty. I worked for two
years there, and I focused on performance optimization of large scale finite element simulations of 
hydraulic systems. During this time, I also conducted work for my graduate thesis about the design
of an embedded multi-core Java processor.\\
Using vectorization, OpenMP, and MPI programming techniques I optimized the running time of an 
hydrodynamical model which was used by civil engineers to predict the effect of constructing a bridge
over the Paran\'a
\footnote{The Paran\'a is the second-longest river in South America, and the mentioned bridge
will be the biggest bridge in the country once it's construction is finished during this year. Please
check my resume for a link to a publication describing this work} River,
in Argentina. \\
My Master thesis solved the problem of building a Java Virtual Machine as a multicore system. At that time
multicore processors were an emerging technology so my contribution was to propose solutions to 
problems such as Scheduling of Java threads, Synchronization, and Garbage collection across multiple
processors. \\
After completing my Masters thesis I left the Computer Architecture Laboratory and pursued a career
in software development in industry. The first position I held was in Motorola Argentina, as an embedded
software developer. I focused on software for communication systems such as Soft Switches that are built
as a cluster of special purpose boards. I acquired valuable experience in software development in general,
working with big amounts of code, collaborating with a team of people and planing work to meet deadlines.
During this same period of time, I started to pursue a degree in Distributed Systems at the Faculty of 
Mathematics at National University of C\'ordoba.
\\
After completing my second degree and acquiring more experience as an engineer I decide to look
for positions more challenging on the technical side. The work at Motorola had become too tedious
in the sense that too much attention was put by management on artificial quality goals rather
that in actually trying to create better products.
\\
In my search for real engineering challenges I joined Intel. Projects at Intel are truly ambitious,
and they really focus on delivering value. My work at Intel has been mainly on Context Awareness and 
Recommender Systems. The Context Awareness project was about providing APIs to developers of mobile apps
so they can access high level states such as the physical activity or the user 
location\footnote{The high level location would be a value in the set \{home, work, other\}, and the low
level raw sensor data would include for example the GPS coordinates}.
This involved the use of several Machine Learning techniques to perform mining of raw sensor data such
as accelerometers, gps, cameras and microphones and to actually deliver value to the the developer.\\
My work in Recommender Systems focused on the development of a large scale graph processing database which 
was used as a Knowledge Representation engine that allows the real time processing of user recommendations in 
the form of graph queries. In this context, nodes in the graph would represent consumer items such as books or 
songs and arcs in the graph would represent relationships among these items. I worked both developing the 
infrastructure for this database and computing the relationships between nodes. I'm proud to say
that we used really cutting edge technology both in the infrastructure\footnote{We used Akka, and Scala
which are among the top choices when developing large distributed systems.} and in the data mining part as well.
The final product was a really competitive query processing engine that performs better than competitors.
\\
I thank you for taking the time to consider my application and I look forward to the opportunity of further
discussing with you details of the Vertica platform. Should you have any question about my previous experience,
please do not hesitate to contact me.
\\

Sincerely,
\\
\\
Jose P. Alberto Andreotti.

\end{document}




