%\documentstyle{article}
\documentclass[letter,12pt,english]{article}
\special{papersize=8.5in,12in}

\usepackage{ifsym}
\usepackage[dvipdfm, hypertex]{hyperref}
\oddsidemargin=.15in
\evensidemargin=.15in
\textwidth=6in
\topmargin=-.5in
\textheight=9in
\parindent=0in
\pagestyle{plain}


\begin{document}




{ \hfill \today \\ \\ 
Prof. Guojun Lu\\
Gippsland Campus\\
FedUni\\
\\
\\

\large
\bf Research proposal} \\*[-.8pc]
\\
\\

This document describes a research proposal for the project "Image Search and Retrieval" in charge of Prof. Guojun Lu
at FedUni, Australia. This document is divided in two parts; the first describing a potential research path for
the algorithmic problems and a second one describing a proposal for an infrastructure to efficiently perform queries
on large image datasets.\\\\
{\bf Image Retrieval}\\\\
Approaches to image information retrieval based solely on natural language descriptions of the images not only suffer
from the problem that such descriptions may not always be available but also the process of matching images according
to user descriptions may lead to undesired results. Users may produce different statements even for the same image
according to their mood, the context of the image or other factors that are hard to capture without the
intervention of a human. An effort to overcome this high variability in results obtained by crowd sourcing text 
description of images is explored by Socher\cite{socher} et al., where the authors propose a method to segment,
annotate and classify images based on deep architectures.\\
Besides outperforming state of the art methods in similar image tasks, deep architectures provide also a means
of learning hierarchical
levels of features with each level providing a further sophistication in the abstraction used to describe the data.
In this way, detected parts of the image in low layers can be recursively composed in higher layers to convey a 
higher level of representation. These abstractions can be used to better match one image to another.\\
Although this approach removes the problems of relying only on low level feature to understand an image, there's
still an improvement that could potentially be applied to the solution.\\
Deep architectures are good learning high level features over low level ones in an unsupervised manner. However,
they still require labelling of the data to be trained successfully. To overcome this problem one may think of latent
variable methods such as the one proposed in\cite{lda} for natural language documents. In these models, the different
topics in which documents are classified are learnt as part of the training process, not depending on hand labelled
data. Then the similarity between two documents can be assessed by computing comparing how much each document
belongs to a topic class.\\
So, a desiderata for a method in image similarity would include the hierarchical abstraction power of deep
architectures, the unsupervised learning capabilities of latent variable models, and a method to retrieve
images fast. A really promising approach has been proposed by Hinton and Krizhevsky\cite{hinton} where the
authors not only propose a means to address image similarity in an unsupervised fashion, but also a way of
retrieving similar images using semantic hashing\cite{hinton1}.\\
In this approach, many levels of features are learnt, and these features are used to initialize autoencoders.
The autoencoders are then used to map each image to a binary word. Similar images map to similar binary words.
To retrieve images that are similar to a query image, the binary code is computed for the query image, and
small modifications are performed in the code to obtain the addresses of the similar images.\\
Another approach to image similarity would be to train deep neural networks to classify objects and then use
the activity vectors from the last hidden layer to compare between images. Although this leads to excellent
results as can be seen in \cite{hinton2}, it is a supervised method, and does not have the advantages from
semantic hashing.\\
Another related problem is how to include the captions in the retrieval process when they are available, 
how to decide when the captions are relevant, and what to do when captions are missing. To have an idea
of how complex this problem can be we can use figure where we see two images with the same objects but
different captions and meaning.
Explore how captions can be used in a query. Add caption to query image.\\

Still a method that meets all the above mentioned requirements needs to be found, and needs to be tested
in an experimental environment to asses for generalization in new datasets and suitability in adapting
to practical computing infrastructures.} among them.\\


\bf Infrastructure} \\*[-.8pc]
\\
\\

Although the idea of semantic hashing for image retrieval is a great contribution because it allows for constant
time retrieval of similar images independently of the database size, more research must be done to study how
these techniques can handle a wider variety of use cases beyond matching objects and also on how they can be
implemented in a production environment where large scale retrieval, perhaps in real time, must be provided.\\
Graph databases.




  \begin{thebibliography}{1}

  \bibitem{socher} Parsing Natural Scenes and Natural Language with Recursive Neural Networks, Richard Socher,
  Cliff Lin, Andrew Y. Ng, and Christopher D. Manning. 

  \bibitem{lda} Probabilistic topic models. Steyvers, M. \& Griffiths, T. (2007).

  \bibitem{hinton1} Using Very Deep Autoencoders for Content-Based Image Retrieval. Krizhevsky, A. and Hinton, G.E.(2011). 
  
  \bibitem{hinton2} ImageNet Classification with Deep Convolutional Neural Networks. Krizhevsky, A. and Hinton, G.E.(2011). 
  
  \end{thebibliography}


\end{document}




