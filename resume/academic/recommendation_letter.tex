%\documentstyle{article}
\documentclass[letter,12pt,english]{article}
\special{papersize=8.5in,12in}

\usepackage{ifsym}
\usepackage[dvipdfm, hypertex]{hyperref}
\oddsidemargin=.15in
\evensidemargin=.15in
\textwidth=6in
\topmargin=-.5in
\textheight=9in
\parindent=0in
\pagestyle{plain}


\begin{document}

{
\raggedleft{}
Dr. Nicol\'as Wolovick \\
FaMAF, National University of C\'ordoba \\
October, 2014 \\
}

{ 

\large
\bf Recommendation Letter for Alberto Andreotti} \\*[-.8pc]
\\

The first time I met Alberto was in 2008 at a Postgraduate Programme in Distributed Systems at the Faculty of Mathematics, Astronomy and Physics (FaMAF) of the National University of Córdoba (UNC). I was his professor in "Concurrent Programming in Java" one of the courses of the programme.
The topics of this course were divided into theoretical foundations of concurrent programming and practical activities in the Java language. Besides being one of my best students, Alberto showed, not only a really good understanding of even the most complex topics from both parts, but he could also relate theoretical methods to day-to-day programming practices. That is to me a clear sign of intellectual maturity. The aforementioned gets emphasized by the fact his undergraduate background was weak in both theory and practice of concurrency.

In 2010 he also took the course on "High Performance Computing" that is taught at FaMAF in coordination with CSC7600 of Louisiana State University (LSU). There he also succeeded in getting a deep understanding of all the topics involved and the capability to perform practical work as well. He was also a top student in that class.
\\
During 2011, he addressed me to be his advisor in the final work of the Specialization Programme in Distributed Systems. We delve into the area of HPC, but we wanted to make some novel move. The basic idea was to apply the map-reduce frameworks like Hadoop to classical problems of HPC. Once again Alberto did it from scratch: functional programming, highly distributed and fault tolerance systems, complex configurations, etc. He obtained in a very short time (40 hours) two proof-of-concept solutions, measurements in some clusters, besides writing a very good report of his work.
\\
The fact that he accomplished all these academic goals at the same time he is working full time in the industry, makes it even more remarkable. All that would be impossible without a big amount of talent, patience and the desire to stay up to date with the newest techniques available.
\\
I believe Alberto will play an excellent part in any further educational endeavour that he decides to take. And I specially recommend him for any computer science related programme.
\\
\\
Sincerely, 
\\
\\
Dr. Nicol\'as Wolovick
\end{document}




