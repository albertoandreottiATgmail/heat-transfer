%\documentstyle{article}
\documentclass[a4paper,11pt,english]{article}
\usepackage{ifsym}
\usepackage[dvipdfm, hypertex]{hyperref}
\oddsidemargin=.15in
\evensidemargin=.15in
\textwidth=6in
\topmargin=-.5in
\textheight=9in
\parindent=0in
\pagestyle{empty}
%Conditional compilation statement for professional version.
\usepackage{ifthen}
\newboolean{DEBUG}
\setboolean{DEBUG}{true}

\begin{document}

\begin{center}
{\Large Jose Pablo Alberto Andreotti} \\[.5pc]
(54) 351-4730292 $\;$ albertoandreotti\verb|@|gmail.com $\;$ (54) 351-156526363 \\[3pc]
\end{center}
{\large \bf Informaci\'on Personal } \\*[-.8pc]
\underline{\hspace{6in}} \\
\\
{\bf Nacido en Santa Cruz, Argentina}, 08/10/1982.\\
\\
{\bf Nationalidad}, Argentina/Italiana.\\
\\
{\bf Direcci\'on}, Emilio Castelar 552, C\'ordoba. Argentina.\\

{\large \bf Educaci\'on} \\*[-.8pc]
\underline{\hspace{6in}} \\
\\
{\bf Especialista en Sistemas y Servicios Distribuidos}\\
\href{http://www.famaf.unc.edu.ar/}{Famaf}, Universidad de C\'ordoba, 2011. \\
Promedio: 9.22/10. \\
\\
{\bf Ingeniero en Computaci\'on}\\
Facultad de Ciencias Exactas F\'isicas y Naturales, Universidad de C\'ordoba, 2007.\\
Tesis: Construcci\'on de una JVM dual core para sistemas embebidos de tiempo real.\\
Promedio: 8.14/10. \\
\\
{\bf Secundario, Colegio Coraz\'on de Mar\'ia}\\
Bachiller Comercial, 2000.\\
Promedio: 8.22/10.\\
\\
{\bf Tesis} \\
\href{http://www.jopdesign.com}{JOP}(Java Optimizad Processor) es una implementaci\'on hardware de la JVM.
Mi tesis comprende el dise\~{n}o y la implementaci\'on de una JVM multiprocesador utilizando JOP.\\

\ifthenelse {\boolean{DEBUG}}
{
{\large \bf Experiencia en Programaci\'on} \\*[-.8pc]
\underline{\hspace{6in}} \\
$\triangleright$ Desarrollo de Software embebido en C/C++, and Java.\\
$\triangleright$ Mantenimiento de Software Linux en C/C++.\\
$\triangleright$ Experiencia en plataformas m\'oviles como Android e iOS.\\
$\triangleright$ Test Development en Bash/Perl/Python/Java.\\
$\triangleright$ Experiencia en desarrollo orientado a objetos.\\
\\
}

 \\
{\large \bf Cursos de Postgrado}\\ *[-.8pc]
\underline{\hspace{6in}} 
$\triangleright$ High Performance Computing: Models, Methods and Means\footnote{Curso no estructurado de postgrado
, Fa.M.A.F. Sincronizado con el Prof. Thomas Sterling.
\href{http://www.cct.lsu.edu/csc7600/Home.html}{CSC7600}.}, primer semestre 2010.\\
$\triangleright$ La Web sem\'antica, Doctor Jorge Cardoso, SAP Research Alemania, primer semestre de 2009.\\
$\triangleright$ Introducci\'on al procesamiento de la senal radar, Msc Oscar Bria, 
\href{http://www.invap.net/index-e.php}{INVAP}, segundo semestre de 2009. \href{http://postgrado.info.unlp.edu.ar/Carrera/Programas/Contenidos_IPSR.pdf}{Programa}.\\
$\triangleright$ Performance and Scaling in E-Commerce Systems WS, Alex Buchmann, segundo semestre de 2008. 
\href{http://www.dvs.tu-darmstadt.de/teaching/perf/2008/}{Programa}.\\


{\large \bf Experiencia Profesional }\\*[-.8pc]
\underline{\hspace{6in}}
\\
\\
\begin{tabular}{ p{2cm} l }
  {\bf Actual}  & {\bf Sr. Software Engineer en \href{http://www.intel.com}{Intel}.}\\ 
				
  {\bf Pasado}& 	{\bf Software Engineer en \href{http://www.nimbuzz.com/en/about}{Nimbuzz}.}\\
			& 	{\bf Miembro del Consejo de Ingenier\'ia en Computaci\'on en } \\
			& 	{\bf la Universidad Nacional de C\'ordoba.}\\ \\
			& 	{\bf Software Engineer en Motorola Argentina}\\
			& 	{\bf Trabajo independiente en desarrollo de software y mantenimiento.}\\
			&	Desde 2007. Trabaj\'e en proyectos para la optimizaci\'on de aplicaciones\\
			& 	en sistemas multiprocesador.\\
			& 	{\bf Laboratorio de Arquitectura de Computadoras, UNC}\\
			& 	Pasant\'ia de Investigaci\'on, Desde Marzo/2006 hasta Octubre/2007.\\
			& 	$\triangleright$ Optimizaci\'on de aplicaciones cient\'ificas sobre arquitecturas multiprocesador\\
			& 	utilizando MPI, HPF y OpenMP.\\
			& 	$\triangleright$ Testing de sistemas embebidos para aplicaciones de electro medicina.\\
			& 	{\bf Testing del procesador JOP y JDK}\\
			& 	Trabaj\'e durante 2007 para Martin Schoeberl de la Universidad Tecnol\'ogica de\\
			&	Viena, Austria.\\
			& 	{\bf Optimizaci\'on de Software Cient\'ifico.}\\
			& 	Trabaj\'e para Gerardo Hillman del Laboratorio de Recursos Hidr\'icos de la \\
			& 	Universidad Nacional de C\'ordoba, Argentina.\\
				
\end{tabular} 
\\\\
{\bf Publications} \\*[-.8pc]
\underline{\hspace{6in}} \\
$\triangleright$ Using JOP to build a chip multiprocessor JVM for embedded realtime systems. Annals of CACIC, 2007.\\
%$\triangleright$ Optimization of a bidimensional hydrodynamic model. ENIEF, 2009.\\
$\triangleright$ Optimization of an hydrodynamical bidimensional model. \href{http://enief2009.pladema.net/index.html}{ENIEF 2009}. \\


{\bf Becas y Premios} \\*[-.8pc]
\underline{\hspace{6in}} \\
%\begin{itemize}
%\renewcommand{\labelitemi}{$\star$}
%\item
$\triangleright$ Beca concedidad para completar el trabajo final de carrera.
Otorgada por la Agencia C\'ordoba Ciencia, 2006.\\ 
%\item
$\triangleright$ Distinci\'on al mejor promedio de la Especialidad.
Otorgado por el Colegio de Ingenieros Especialistas de C\'ordoba, 2007.\\
%\end{itemize}

{\large \bf Idiomas} \\*[-.8pc]
\underline{\hspace{6in}} \\

{\bf Espa\~{n}ol}\\
$\triangleright$ Nativo.
\\

{\bf Alem\'an}\\
$\triangleright$ Completados los dos primeros a\~{n}os en el Departamento Cultural de la Facultad de Lenguas, UNC. 

 \\

{
{\bf Ingl\'es}\\
$\triangleright$ TOEFL iBT. Calificaci\'on 110/120, Marzo 2008.\\
$\triangleright$ First Certificate Examination grade B. Otorgado por la  Cambridge University, 2000.\\
$\triangleright$ Varios cursos tomados desde 1992 hasta 2000.\\


{\bf Italiano}\\
$\triangleright$ Completados los cuatro a\~{n}os en el Departamento Cultural de la Facultad de Lenguas, UNC. 
\\

{\bf Franc\'es}\\
$\triangleright$ Curso de un a\~{n}o en "Alliance Fran\c caise C\'ordoba", 2000.
\\



{\large \bf Experiencia Docente} \\*[-.8pc]
\underline{\hspace{6in}} \\
{\bf Programaci\'on Concurrente}\\
Facultad de Ingenier\'ia, 2004.
\\
{\bf Ayudante alumno en An\'alisis Matem\'atico II}\\
Escuela de Ingenier\'ia Electr\'onica y Computaci\'on, 2003.
\\

{\large \bf Participaci\'on como expositor en charlas y conferencias} \\*[-.8pc]
\underline{\hspace{6in}} \\
{\bf  ENIEF, Congress on Numerical Methods and their Applications, 2009.}\\
T\'itulo: Optimization of an hydrodynamical bidimensional model. 
\\
{\bf  Segundas conferencias abiertas en Ingenier\'ia en Computaci\'on Univesidad Nacional de C\'ordoba.}\\
T\'itulo: Programaci\'on Paralela.
\\
{\bf  Primeras conferencias abiertas en Ingenier\'ia en Computaci\'on Univesidad Nacional de C\'ordoba.}\\
T\'itulo: A multiprocessor JVM based on JOP.
\\
{{\bf  XIII Argentinian Congress on Computer Science}\\
T\'itulo: Using JOP to build a chip multiprocessor JVM for embedded real-time systems.}
\\

{\large \bf Tiempo Libre} \\*[-.8pc]
\underline{\hspace{6in}} \\
$\triangleright$ Trekking.\\
$\triangleright$ Guitarrista amateur.\\
\\
\end{document}




