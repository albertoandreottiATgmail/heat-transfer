%\documentstyle{article}
\documentclass[a4paper,11pt,english]{article}
\usepackage{ifsym}
\usepackage[dvipdfm, hypertex]{hyperref}
\oddsidemargin=.15in
\evensidemargin=.15in
\textwidth=6in
\topmargin=-.5in
\textheight=9in
\parindent=0in
\pagestyle{empty}
%Conditional compilation statement for professional version.
\usepackage{ifthen}
\newboolean{DEBUG}
\setboolean{DEBUG}{false}

\begin{document}

\begin{center}
{\Large Jose Pablo Alberto Andreotti} \\[.5pc]
(54) 351-4809267 $\;$ albertoandreotti\verb|@|gmail.com $\;$ (54) 351-155937792 \\[3pc]
\end{center}
{\large \bf Informaci\'on Personal } \\*[-.8pc]
\underline{\hspace{6in}} \\
\\
{\bf Nacido en Santa Cruz, Argentina}, 08/10/1982.\\
\\
{\bf Nationalidad}, Argentina/Italiana.\\
\\
{\bf Direcci\'on}, Pasaje Cornet 1922, C\'ordoba. Argentina.\\

{\large \bf Educaci\'on} \\*[-.8pc]
\underline{\hspace{6in}} \\
\\
{\bf Maestr\'ia en Ciencias de la Computaci\'on, Aprendizaje Autom\'atico.}\\
\href{http://www.dirinfo.unsl.edu.ar/posgrado/}{Facultad de Inform\'atica}, Universidad Nacional de San Luis, esperado 2018. \\
\\
{\bf Especialista en Sistemas y Servicios Distribuidos}\\
\href{http://www.famaf.unc.edu.ar/}{Famaf}, Universidad de C\'ordoba, 2011. \\
Promedio: 9.22/10. \\
\\
{\bf Ingeniero en Computaci\'on}\\
Facultad de Ciencias Exactas F\'isicas y Naturales, Universidad de C\'ordoba, 2007.\\
Tesis: Construcci\'on de una JVM dual core para sistemas embebidos de tiempo real.\\
Promedio: 8.14/10. \\
\\
{\bf Secundario, Colegio Coraz\'on de Mar\'ia}\\
Bachiller Comercial, 2000.\\
Promedio: 8.22/10.\\
\\
{\bf Tesis} \\
\href{http://www.jopdesign.com}{JOP}(Java Optimizad Processor) es una implementaci\'on hardware de la JVM.
Mi tesis comprende el dise\~{n}o y la implementaci\'on de una JVM multiprocesador utilizando JOP.\\
\\
{\large \bf Cursos de Postgrado}\\ *[-.8pc]
\underline{\hspace{6in}}\\
\\
$\triangleright$ \href{http://www.famaf.unc.edu.ar/~jsanchez/aavc16/}{Aprendizaje Autom\'atico en Visi\'on por Computadoras}, Jorge Sanchez, Fa.M.A.F, Universidad Nacional de C\'ordoba, 2016.\\
$\triangleright$ \href{http://cs.famaf.unc.edu.ar/wiki/materias/pln}{Procesamiento del Lenguaje Natural y Miner\'ia de Texto}, Franco Luque y Laura Alonso, Fa.M.A.F, Universidad Nacional de C\'ordoba, 2015.\\
$\triangleright$ \href{http://aprendizajengrande.net/}{Aprendizaje Autm\'atico sobre grandes vol\'umenes de Datos},  \href{http://duboue.net/}{Pablo Duboue}, Fa.M.A.F, Universidad Nacional de C\'ordoba,  2014.\\
$\triangleright$ Introduction to Data Science, University of Washington Coursera. 2014\\
$\triangleright$ \href{https://www.coursera.org/course/progfun}{Functional Programming Principles in Scala},  Martin Odersky. Ecole Polytechnique Federale de Lausanne, December of 2013.  \\
$\triangleright$ \href{https://www.coursera.org/course/images}{Image and video processing: From Mars to Hollywood with a stop at the hospital}, Guillermo Sapiro, Duke University, 2013. \\
$\triangleright$ \href{https://www.coursera.org/course/dsp}{Digital Signal Processing}, Paolo Prandoni and Martin Vetterli, Ecole Polytechnique Federale de Lausanne, 2013.\\
$\triangleright$ \href{http://www.ml-class.org}{Machine Learning \footnote{Versi\'on online del curso.}}., Andrew Ng, Stanford University, 2012. \\
$\triangleright$ High Performance Computing: Models, Methods and Means\footnote{Curso no estructurado de postgrado
, Fa.M.A.F. Sincronizado con el Prof. Thomas Sterling.
\href{http://www.cct.lsu.edu/csc7600/Home.html}{CSC7600}.}, primer semestre 2010.\\
$\triangleright$ La Web sem\'antica, Doctor Jorge Cardoso, SAP Research Alemania, primer semestre de 2009.\\
$\triangleright$ Introducci\'on al procesamiento de la señal radar, Msc Oscar Bria, 
\href{http://www.invap.net/index-e.php}{INVAP}, segundo semestre de 2009. \href{http://postgrado.info.unlp.edu.ar/Carrera/Programas/Contenidos_IPSR.pdf}{Programa}.\\
$\triangleright$ Performance and Scaling in E-Commerce Systems WS, Alex Buchmann, segundo semestre de 2008. 
\href{http://www.dvs.tu-darmstadt.de/teaching/perf/2008/}{Programa}.\\


{\large \bf Experiencia Profesional}\\*[-.8pc]
\underline{\hspace{6in}}
\\
\begin{tabular}{ p{2cm} l }

  {\bf Actual} & {\bf White Prompt, Sr. Software Engineer.}\\
   En./2016   & $\triangleright$ Desarrollo de Sistemas para procesar grandes vol\'umenes de datos, \\ 
   Presente    & arquitecturas de cluster, microservicios, programaci\'on orientada\\
                    & a objetos y funcional. \\
                
  \\

  {\bf Previo } & {\bf Samsung, Sr. Software Engineer.}\\
   Jun./2014   & $\triangleright$ Trabajo para el Samsung Strategy and Innovation Center (SSIC)\\
   Dic./2015     & construyendo la plataforma para Big Data SAMI.\\
                & $\triangleright$ Sr. Software Engineer con foco en escalabilidad, performance,\\
                & tiempo-real e IoT(Internet Of Things).\\ 
  \\
                
                    & {\bf \href{http://www.intel.com}{Intel Corp}, Sr. Software Engineer.}\\
                      & Contribu\'i de manera decisiva para en los siguientes proyectos,\\
   En./2011     & $\triangleright$  \href{https://software.intel.com/en-us/context-sensing-sdk}{Context Awareness and Inference Engine}.\\
   Jun./2014     & $\triangleright$ Base de Datos basada en Grafos para representaci\'on del Conocimiento. \\
                & con aplicaciones en Recomendaciones personalizadas.\\
                & $\triangleright$ Colabor\'e con CSP, la plataforma de Cloud Services de Intel.\\
\\     
   Jun./2014     & {\bf \href{http://www.nimbuzz.com/en/about}{Nimbuzz}, Software Engineer.}\\
   En./2011   	&  $\triangleright$ Desarrollo en iPhone/iOS del cliente VoIP Nimbuzz.\\ 
\\
   2011         & {\bf Miembro del Concejo de Ingenier\'ia en Computaci\'on} \\
		& {\bf Universidad Nacional de C\'ordoba.}\\ 
		& $\triangleright$ Revisi\'on de los planes de estudio y la curricula, acreditaci\'on de la carrera.\\
\\
   Jun. 2008     & {\bf Ingeniero de Software en Motorola Argentina}\\
   Jun. 2010     & $\triangleright$ Desarrollo de Software para Sistemas Embebidos en C/C++ para \\
                      &  productos de Cable Modem.\\
		& $\triangleright$ Desarrollo de Tests Funcionales para la \href{http://www.motorola.com/web/Business/_Documents/White%20Paper/_Static%20files/NBBS%20WiMAX%20White%20Paper%20557127-001-b.pdf}{plataforma NBBS}.\\
		& $\triangleright$ Developed and maintained \href{http://en.wikipedia.org/wiki/Advanced_Telecommunications_Computing_Architecture}{AdvancedTCA} clusters running Linux.\\

\\
\end{tabular}
\begin{tabular}{ p{2cm} l }
   2007/2008    & {\bf Trabajo Independiente}\\
		& Optimizaci\'on de aplicaciones cient\'ificas mediante el uso de t\'ecnicas de \\
                & HPC.\\
   Mzo./2006    & {\bf Laboratorio de Arquitectura de Computadoras, UNC}\\
   Oct./2007    & Pasant\'ia de Investigaci\'on.\\
		&$\triangleright$ Optimizaci\'n de aplicaciones Cient\'ificas mediante t\'ecnicas de\\
		& programaci\'on paralela  con MPI, HPF y OpenMP.\\
		&$\triangleright$ Desarrollo de Pruebas Funcionales para equipos de electromedicina.\\
\\
   2007         &{\bf Unit Testing del Procesador JOP}\\
		& Trabaj\'e creando casos de prueba para \href{http://jopdesign.com}{Java Optimized Processor (JOP)}.
\\
					
\end{tabular}
\\
\\
{\large \bf Experiencia en Programaci\'on} \\*[-.8pc]
\underline{\hspace{6in}} \\
$\triangleright$ Desarrollo en Scala/Akka, Spark, AWS.\\
$\triangleright$ Desarrollo en Python utilizando principalmente el stack de Machine Learning(Scikit-learn, NLTK, Gensim, Scikit-image, OpenCV.)\\
$\triangleright$ Experiencia en desarrollo orientado a objetos y funcional.\\
$\triangleright$ Experiencia en plataformas m\'oviles como Android e iOS.\\
$\triangleright$ Desarrollo de Software embebido en C/C++, y Java.\\
$\triangleright$ Mantenimiento de Software Linux en C/C++.\\
$\triangleright$ Test Development en Bash/Perl/Python/Java.\\
\\
{\bf Publicaciones} \\*[-.8pc]
\underline{\hspace{6in}} \\
$\triangleright$ Using JOP to build a chip multiprocessor JVM for embedded realtime systems. Annals of CACIC, 2007.\\
%$\triangleright$ Optimization of a bidimensional hydrodynamic model. ENIEF, 2009.\\
$\triangleright$ Optimization of an hydrodynamical bidimensional model. \href{http://enief2009.pladema.net/index.html}{ENIEF 2009}. \\

{\bf Becas y Premios} \\*[-.8pc]
\underline{\hspace{6in}} \\
%\begin{itemize}
%\renewcommand{\labelitemi}{$\star$}
%\item
$\triangleright$ Beca concedidad para completar el trabajo final de carrera.
Otorgada por la Agencia C\'ordoba Ciencia, 2006.\\ 
%\item
$\triangleright$ Distinci\'on al mejor promedio de la Especialidad.
Otorgado por el Colegio de Ingenieros Especialistas de C\'ordoba, 2007.\\
%\end{itemize}

{\large \bf Idiomas} \\*[-.8pc]
\underline{\hspace{6in}}
{\bf Espa\~{n}ol}\\
$\triangleright$ Nativo.\\
{\bf Alem\'an}\\
$\triangleright$ Completados los dos primeros a\~{n}os en el Departamento Cultural de la Facultad de Lenguas, UNC. \\
{\bf Ingl\'es}\\
$\triangleright$ TOEFL iBT. Calificaci\'on 110/120, Marzo 2008.\\
$\triangleright$ First Certificate Examination grade B. Otorgado por la  Cambridge University, 2000.\\
$\triangleright$ Varios cursos tomados desde 1992 hasta 2000.\\
{\bf Italiano}\\
$\triangleright$ Completados los cuatro a\~{n}os en el Departamento Cultural de la Facultad de Lenguas, UNC. 
\\
{\bf Franc\'es}\\
$\triangleright$ Curso de un a\~{n}o en "Alliance Fran\c caise C\'ordoba", 2000.
\\



{\large \bf Experiencia Docente} \\*[-.8pc]
\underline{\hspace{6in}} \\
{\bf Programaci\'on Concurrente}\\
Facultad de Ingenier\'ia, 2004.
\\
{\bf Ayudante alumno en An\'alisis Matem\'atico II}\\
Escuela de Ingenier\'ia Electr\'onica y Computaci\'on, 2003.
\\

{\large \bf Participaci\'on como expositor en charlas y conferencias} \\*[-.8pc]
\underline{\hspace{6in}} \\
{\bf  ENIEF, Congress on Numerical Methods and their Applications, 2009.}\\
T\'itulo: Optimization of an hydrodynamical bidimensional model. 
\\
{\bf  Segundas conferencias abiertas en Ingenier\'ia en Computaci\'on Univesidad Nacional de C\'ordoba.}\\
T\'itulo: Programaci\'on Paralela.
\\
{\bf  Primeras conferencias abiertas en Ingenier\'ia en Computaci\'on Univesidad Nacional de C\'ordoba.}\\
T\'itulo: A multiprocessor JVM based on JOP.
\\
{{\bf  XIII Argentinian Congress on Computer Science}\\
T\'itulo: Using JOP to build a chip multiprocessor JVM for embedded real-time systems.}
\\

{\large \bf Tiempo Libre} \\*[-.8pc]
\underline{\hspace{6in}} \\
$\triangleright$ Trekking.\\
$\triangleright$ Guitarrista amateur.\\
\\
\end{document}




